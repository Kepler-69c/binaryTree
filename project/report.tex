\documentclass[a4paper,11pt]{article}
\usepackage[total={148mm,230mm}]{geometry}
\usepackage[T1]{fontenc}
\usepackage[utf8]{inputenc}
%\usepackage{lmodern}
\usepackage{tikz}
\usepackage{listings}
\usepackage[english,ngerman]{babel}
\usepackage{graphicx}
\usepackage{xcolor}
\usepackage{caption}
\usepackage{apacite}

\definecolor{codegreen}{rgb}{0,0.6,0}
\definecolor{codegray}{rgb}{0.5,0.5,0.5}
\definecolor{codepurple}{rgb}{0.58,0,0.82}
\definecolor{backcolour}{rgb}{0.95,0.95,0.92}

\lstdefinestyle{cplusplus}{
    backgroundcolor=\color{backcolour},   
    commentstyle=\color{codegreen},
    keywordstyle=\color{magenta},
    numberstyle=\tiny\color{codegray},
    stringstyle=\color{codepurple},
    basicstyle=\ttfamily\footnotesize,
    breakatwhitespace=false,         
    breaklines=true,                 
    captionpos=b,                    
    keepspaces=true,                 
    numbers=left,                    
    numbersep=5pt,                  
    showspaces=false,                
    showstringspaces=false,
    showtabs=false,                  
    tabsize=2
}

\lstset{style=cplusplus}

\newcommand{\CC}{C\nolinebreak\hspace{-.05em}\raisebox{.4ex}{\tiny\bf +}\nolinebreak\hspace{-.10em}\raisebox{.4ex}{\tiny\bf +}\hspace{.1em}}

\title{Vergleich der Komplexität von \break Tree-Search-Algorithmen}
\author{}


\begin{document}

\maketitle
\thispagestyle{empty}
\pagebreak
\selectlanguage{english}
\begin{abstract}
In diesem Projekt wird die Komplexität von verschiedenen einfachen Tree-Search-Algorithmen untersucht. Es werden verschiedene Algorithmen wie die Tiefensuche, Breitensuche und andere heuristische Suchalgorithmen auf binären Suchbäumen getestet. Dabei wird die Anzahl der durchsuchten Knoten und die benötigte Zeit gemessen, um die Effizienz der Algorithmen zu vergleichen. Das Ziel ist es, die Komplexität der Suchalgorithmen zu berechnen, um diese dadurch zu vergleichen.
\end{abstract}
\selectlanguage{ngerman}
\pagebreak
\tableofcontents
\pagebreak

%TODO: aus der Sicht einer 3. Person / Passiv schreiben!!
%TODO: alle begriffe, die oft vorkommen, gleich schreiben: z.B. Tree-Search Algorithmus (angang immer gross, verbunden)
%TODO: es ist kein suchbaum, den wir benutzt haben, sondern ein normaler baum -> ändern

\section{Einleitung}
Das Ziel dieses Projekts ist es, die Komplexitäten verschiedener Tree-Search-Algorithmen hinsichtlich der Größe und Tiefe des Baumes zu vergleichen. Dabei wird ein Programm in der Programmiersprache \CC erstellt, um die Berechnungen zu messen und zu überprüfen. 

Tree-Search-Algorithmen sind in der künstlichen Intelligenz weit verbreitet und werden verwendet, um Entscheidungsprobleme zu lösen. Da verschiedene Algorithmen unterschiedliche Vorteile und Nachteile aufweisen, ist es wichtig, ihre Leistungsfähigkeit und Komplexität zu verstehen. Eine Art, dies zu messen ist die Einschätzung der Komplexität. Diese ist eine Abschätzung für das obere Limit der Laufzeit eines Tree-Search-Algorithmus bei einer Anzahl von $n$ und einer Tiefe von $d$ Knoten. In diesem Projekt wird die Komplexität von verschiedenen Tree-Search-Algorithmen miteinander verglichen, um eine bessere Einschätzung der Komplexität bei der Verwendung dieser Algorithmen zu ermöglichen.


\section{Theorie}
\subsection{Suchbaum}
Suchbäume sind Datenstrukturen aus der Graphentheorie. Der Begriff 'Suchbaum' bezieht sich darauf, dass die Datenstruktur baumartig organisiert ist. Ein Baum besteht aus einer Wurzel, die den obersten Knoten (engl. nodes) darstellt, und einer Anzahl von Zweigen oder Kanten, die von der Wurzel zu den Blättern führen, die die Endpunkte des Baums darstellen. Jeder Knoten im Baum repräsentiert eine Entscheidung oder einen Zustand, und jeder Zweig repräsentiert eine Aktion, die von diesem Zustand aus ausgeführt werden kann.

%TODO: package für anführungszeichen hinzufügen

Ein Baum ist also ein zusammenhängender Graph, der keine geschlossenen Pfade enthält \cite{graphs}, d.h. der sich nur in eine Richtung von der Wurzel aus erstreckt.

\subsubsection{binärer Suchbaum}
Ein binärer Suchbaum sind eine besondere Art von Suchbaum, bei dem jeder Knoten maximal zwei Kindknoten (engl. child nodes) haben kann. Ausserdem erfordert ein binärer Suchbaum, im Gegensatz zu einem normalen Suchbaum, eine bestimmte Anordnung der Knoten: Diese sind so angeordnet, dass jeder Knoten kleiner als alle Knoten im rechten Teilbaum ist, aber grösser als alle Knoten im linken Teilbaum ist \cite{c2_algorithms}.
%TODO: letzten Satz besser formulieren / restliche Theorie ist im Plural geschrieben, hier auch ändern? -> nach oder vor Punkt Zitieren?

Der Vorteil dieser speziellen Anordnung ist, dass man schnell einen Knoten im Baum finden kann. Auch ist das Einfügen und Löschen von Knoten aus dem Baum einfacher.

\subsection{Tree-Search Algorithmus}
Tree-Search-Algorithmen sind eine Art von Algorithmen, die zur Lösung von Suchproblemen verwendet werden. Sie sind nützlich, wenn aus einer grossen Anzahl von möglichen Optionen die passende gefunden werden muss.
%TODO: passenden?

Tree-Search-Algorithmen arbeiten mit Suchbäumen, in denen die verschiedenen Optionen als Knoten, die über Kanten verbunden dargestellt sind. Die Suche nach der Lösung des Problems besteht darin, durch die Baumstruktur zu navigieren und den Pfad zu finden, der von der Wurzel (engl. root) bis zum Zielzustand führt.
%TODO: Zustand oder Option?

\subsection{Komplexität von Tree-Search-Algorithmen}
Das Durchsuchen eines Baumes nach einem gewünschten Zustand dauert nicht immer gleich lange. Es hängt vielmehr von einer Vielzahl verschiedener Faktoren ab, wie zum Beispiel der Größe und Tiefe des Suchbaumes oder der Anzahl der Knoten. Eine Abschätzung für die obere Schranke der Suchdauer und damit der Effizienz eines Tree-Search-Algorithmus gibt die Komplexität. 
%TODO: verzweigungsfaktor auch nennen?

Die Komplexität von Tree-Search-Algorithmen wird in der sogenannten $O$-Notation ausgedrückt, die angibt, wie schnell die Laufzeit eines Suchalgorithmus mit der Grösse eines Baumes wächst. Eine Komplexität von $O(n)$ bedeutet beispielsweise, dass die Laufzeit eines Algorithmus linear von der Anzahl der Knoten $n$ abhängt.

\section{Algorithmen im Vergleich}
Untersucht wurden verschiedene Tree-Search-Algorithmen, unter anderem die Tiefensuche (engl. depth-first search, DFS), die Breitensuche (engl. breadth-first search, BFS) und die Monte-Carlo-Baumsuche (engl. Monte Carlo tree search, MCTS). Die Komplexität wurde zunächst für jeden Suchalgorithmus berechnet und anschliessend mit dem Programm in Bezug auf die Anzahl der Baumknoten und die Tiefe, in der das Ergebins lag, überprüft.
%TODO: untersucht -> umformulieren, mathematisch berechnet -> umformulieren

\subsection{Der Suchbaum}
Das Projekt wurde mit einem binären Suchbaum umgesetzt, da diese Baumart den Vorteil hat, leicht skalierbar zu sein. Auch ist die Tiefe des Baumes logarithmisch zu der Anzahl der Knoten, was die Regelung der Baumtiefe erleichtert. Da jeder Knoten zwei \emph{child}-Knoten enthält, ist die Tiefe $d$ eines binären Baumes mit $n$ Knoten
$$d \approx \log_2(n)$$%\text{.}$$
\cite{c2_algorithms}.
%TODO: ist die automatische Tiefe ein Vorteil?
%TODO: stimmt das Zitat?

Der Baum wurde wie folgt umgesetzt: Der Baum besteht aus einzelnen Knoten, die aus einem \emph{struct}-Datentyp bestehen, der über Pointer auf die \emph{child}- und \emph{parent}-Knoten verweist.

\lstinputlisting[language=C++, firstline=16, lastline=23]{../src/BinarySearchTree.h}

Der Vorteil einer solchen Struktur ist, dass einzelne Knoten leicht geändert werden können. Zusätzlich beschleunigt eine solche Implementierung den Aufruf der einzelnen Knoten, was die Gesamtlaufzeit des Programms verkürzt.

\subsection{Breitensuche}

Die Breitensuche (engl. Breadth-First-Search, BFS) ist ein ein Tree-Search Algorithmus, der Graphen schichtweise durchsucht, beginnend bei einem Startknoten und sich schrittweise auf benachbarte Knoten ausbreitet (Abbildung \ref{fig:binary_tree_bfs}).
%TODO: soll das in der Theorie stehen?

\begin{figure}[htbp]
\centering
\begin{tikzpicture}[level distance=1.5cm,  level 1/.style={sibling distance=3cm},  level 2/.style={sibling distance=1.5cm},  every node/.style={circle, draw}]
  \node (A) {A}
    child {node (B) {B}
      child {node (D) {D}}
      child {node (E) {E}}
    }
    child {node (C) {C}
      child {node (F) {F}}
      child {node (G) {G}
        child {node (H) {H}}
      }
    };
  
  \draw[red,very thick] (A)--(B)--(C)--(D)--(E)--(F)--(G)--(H);
\end{tikzpicture}
\caption{Der Breitensuchalgorithmus auf einem binären Baum, rot dargestellt}
\label{fig:binary_tree_bfs}
\end{figure}

Der im Programm gebrauchte Algorithmus benutzte eine \emph{Queue}, also eine Warteschlangen-Datenstruktur für die Speicherung des Baumes. Dabei werden jeweils alle Knoten in einer Schicht nacheinander durchsucht, auf \emph{child}-Knoten untersucht und anschliessend in die \emph{Queue} gespeichert. Dies wiederholt sich bis zum Baumende.

Die schichtweise Suche sorgt dafür, dass Knoten in gleicher Entfernung vom Startknoten zuerst besucht werden, bevor Knoten in weiter entfernten Schichten besucht werden. Dies ist besonders effizient bei flachen Bäumen.
%TODO: rechnung für Geschw.

\subsubsection{Berechnung der Komplexität}

\subsection{Tiefensuche}

Die Tiefensuche (engl. Depth-First-Search, DFS) ist ein Tree-Search Algorithmus der einen Graphen in die Tiefe durchsucht, d.h. sie geht so weit wie möglich in einen Pfad, bevor sie sich umwendet und nach anderen Pfaden sucht (Abbildung \ref{fig:binary_tree_dfs}).

\begin{figure}[htbp]
\centering
\begin{tikzpicture}[level distance=1.5cm,  level 1/.style={sibling distance=3cm},  level 2/.style={sibling distance=1.5cm},  every node/.style={circle, draw}]
  \node (A) {A}
    child {node (B) {B}
      child {node (D) {D}}
      child {node (E) {E}}
    }
    child {node (C) {C}
      child {node (F) {F}}
      child {node (G) {G}
        child {node (H) {H}}
      }
    };
  
  \draw[red,very thick] (A)--(B)--(D)--(E)--(C)--(F)--(G)--(H);
  
\end{tikzpicture}
\caption{Der Tiefensuchalgorithmus auf einem binären Baum, rot dargestellt}
\label{fig:binary_tree_dfs}
\end{figure}

Der für das Projekt geschriebene Algorithmus verwendet eine \emph{stack}-Datenstruktur, um die Reihenfolge der besuchenden Knoten zu speichern. Dabei wird ein Pfad durch den Baum gewählt, der bis zum einem Blatt begangen wird. Danach wird der Algorithmus bis zum letzten nicht vollständig erforschten Knoten zurückgeführt, von wo die Suche nochmals beginnt. Dieser Vorgang wird fortgesetzt, bis alle Knoten besucht wurden.
%TODO: soll stack, queue erklärt werden?
%TODO: soll code gezeigt werden?

\subsubsection{Berechnung der Komplexität}

\subsection{Monte-Carlo Baumsuche}

\section{Resultate}
Der Quellcode für das Programm ist 
\section{Schlussfolgerung}
\section{Reflexion}

\pagebreak
\bibliographystyle{apacite}
\bibliography{bibfile}


\end{document}
